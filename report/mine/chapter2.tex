\chapter{Governing equations and numerical discretization}
\section{Analytical Solution for Fully-Developed Laminar Flow in a Circular Tube}

\subsection{Governing Equations}

The two-dimensional incompressible Navier--Stokes equations in cylindrical coordinates $(r,z)$ for axisymmetric flow are:

\textbf{Continuity:}
\begin{equation}
\frac{1}{r}\frac{\partial}{\partial r}(r v_r) + \frac{\partial v_z}{\partial z} = 0.
\label{eq:continuity_full}
\end{equation}

\textbf{Momentum (radial):}
\begin{equation}
\rho\left( v_r\frac{\partial v_r}{\partial r} + v_z\frac{\partial v_r}{\partial z} \right)
= -\frac{\partial p}{\partial r}
+ \mu\left[\frac{\partial}{\partial r}\left(\frac{\partial v_r}{\partial r}\right)
+ \frac{1}{r}\frac{\partial v_r}{\partial r}
- \frac{v_r}{r^2}
+ \frac{\partial^2 v_r}{\partial z^2}\right].
\label{eq:radial_full}
\end{equation}

\textbf{Momentum (axial):}
\begin{equation}
\rho\left( v_r\frac{\partial v_z}{\partial r} + v_z\frac{\partial v_z}{\partial z} \right)
= -\frac{\partial p}{\partial z}
+ \mu\left[\frac{1}{r}\frac{\partial}{\partial r}\left(r\frac{\partial v_z}{\partial r}\right)
+ \frac{\partial^2 v_z}{\partial z^2}\right].
\label{eq:axial_full}
\end{equation}

\subsection{Assumptions}

For steady, fully-developed, incompressible laminar flow in a horizontal circular tube:

\begin{enumerate}
\item Steady flow: \(\frac{\partial}{\partial t}=0\).
\item Fully-developed: \(\frac{\partial v_z}{\partial z}=0\) and \(v_r = 0\).
\item Axisymmetric flow: \(\frac{\partial}{\partial \theta} = 0\).
\item No-slip at the wall: \(v_z(R)=0\).
\item Gravity acts perpendicular to flow, thus does not appear in \(z\)-momentum.
\end{enumerate}

\subsection{Reduction of the Governing Equations}

With \(v_r = 0\), continuity (\ref{eq:continuity_full}) gives:
\begin{equation}
\frac{\partial v_z}{\partial z} = 0 \quad \Rightarrow \quad v_z = v_z(r).
\label{eq:velocity_only_r}
\end{equation}

The radial momentum equation (\ref{eq:radial_full}) becomes:
\begin{equation}
0 = -\frac{\partial p}{\partial r},
\label{eq:radial_simplified}
\end{equation}
so the pressure depends only on $z$:
\begin{equation}
p = p(z).
\label{eq:pz}
\end{equation}

The axial momentum equation (\ref{eq:axial_full}) simplifies to:
\begin{equation}
\frac{1}{r}\frac{d}{dr}\left(r\frac{dv_z}{dr}\right) = \frac{1}{\mu}\frac{dp}{dz}.
\label{eq:axial_reduced}
\end{equation}

\subsection{Velocity Distribution}

Integrating (\ref{eq:axial_reduced}):
\begin{equation}
\frac{dv_z}{dr} = \frac{1}{2\mu}\frac{dp}{dz}r.
\end{equation}

Integrating again:
\begin{equation}
v_z(r) = \frac{1}{4\mu}\frac{dp}{dz}r^{2} + C_2.
\end{equation}

Applying \(v_z(R)=0\):
\begin{equation}
C_2 = -\frac{1}{4\mu}\frac{dp}{dz}R^{2}.
\end{equation}

Thus the velocity profile is:
\begin{equation}
v_z(r) = \frac{1}{4\mu}\frac{dp}{dz}(r^{2} - R^{2}).
\label{eq:velocity_profile}
\end{equation}

\subsection{Volumetric Flow Rate}

\begin{equation}
Q = \int_0^R v_z(r) (2\pi r)\,dr = -\frac{\pi R^{4}}{8\mu}\frac{\Delta p}{\ell}.
\label{eq:poiseuille}
\end{equation}

\subsection{Mean and Maximum Velocity}

\begin{equation}
V = \frac{Q}{\pi R^{2}} = \frac{R^{2}}{8\mu}\frac{\Delta p}{\ell},
\end{equation}

\begin{equation}
v_{\max} = \frac{R^{2}}{4\mu}\frac{\Delta p}{\ell} = 2V.
\end{equation}

\subsection{Non-Dimensional Velocity Profile}

\begin{equation}
\frac{v_z(r)}{v_{\max}} = 1 - \left(\frac{r}{R}\right)^2.
\label{eq:nondim_profile}
\end{equation}

\subsection{Variable Definitions}

\begin{itemize}
\item $r$: Radial distance from tube center (m).
\item $R$: Tube radius (m).
\item $z$: Axial coordinate (m).
\item $v_z(r)$: Axial velocity as a function of radius (m/s).
\item $v_{\max}$: Maximum velocity at $r = 0$ (m/s).
\item $V$: Mean velocity over cross-section (m/s).
\item $Q$: Volumetric flow rate (m$^{3}$/s).
\item $\Delta p$: Pressure drop along tube length (Pa).
\item $\ell$: Length over which $\Delta p$ acts (m).
\item $\mu$: Dynamic viscosity (Pa·s).
\item $\rho$: Density (kg/m$^{3}$).
\end{itemize}




% 2D incompressible Navier--Stokes and continuity
The two-dimensional incompressible Navier--Stokes equations (Cartesian coordinates \(x,y\)) and continuity are
\begin{align}
\rho\left(\frac{\partial u}{\partial t} + u\frac{\partial u}{\partial x} + v\frac{\partial u}{\partial y}\right)
&= -\frac{\partial p}{\partial x} + \mu\left(\frac{\partial^2 u}{\partial x^2} + \frac{\partial^2 u}{\partial y^2}\right) + \rho f_x,
\label{eq:NS_x}\\[6pt]
\rho\left(\frac{\partial v}{\partial t} + u\frac{\partial v}{\partial x} + v\frac{\partial v}{\partial y}\right)
&= -\frac{\partial p}{\partial y} + \mu\left(\frac{\partial^2 v}{\partial x^2} + \frac{\partial^2 v}{\partial y^2}\right) + \rho f_y,
\label{eq:NS_y}\\[6pt]
\frac{\partial u}{\partial x} + \frac{\partial v}{\partial y} &= 0,
\label{eq:continuity}
\end{align}
where \(u(x,y,t)\) and \(v(x,y,t)\) are the velocity components in the \(x\)- and \(y\)-directions, respectively, \(p(x,y,t)\) is pressure, \(\rho\) is density, \(\mu\) dynamic viscosity, and \(\mathbf{f}=(f_x,f_y)\) body forces per unit mass.

\subsection*{Assumptions (single-line statement)}
Assuming steady (\(\partial/\partial t=0\)), unidirectional fully-developed flow in the \(x\)-direction (\(v\equiv0\), \(u=u(y)\), \(\partial u/\partial x=0\)), negligible body forces in \(x\) (or included in pressure gradient), and constant viscosity and density; under these assumptions all terms in \eqref{eq:NS_x} except the pressure gradient and the viscous term in \(y\) vanish.

Using those assumptions, equation \eqref{eq:NS_x} reduces to the ordinary differential equation
\begin{equation}
0 = -\frac{\mathrm{d}p}{\mathrm{d}x} + \mu \frac{\mathrm{d}^2 u}{\mathrm{d}y^2},
\label{eq:reduced}
\end{equation}
while \eqref{eq:NS_y} gives hydrostatic balance (or \(\partial p/\partial y=0\) if gravity absent), and continuity is satisfied trivially by \(v=0\) and \(\partial u/\partial x=0\).

\section{Solution for flow between parallel plates (\(y=\pm h/2\))}

Integrate \eqref{eq:reduced} twice with respect to \(y\). Write the constant streamwise pressure gradient as \(G\equiv \mathrm{d}p/\mathrm{d}x\) (note \(G\) is negative for flow in positive \(x\) if pressure decreases along \(x\)):
\begin{align}
\mu \frac{\mathrm{d}^2 u}{\mathrm{d}y^2} &= G,
\label{eq:ode1}\\[4pt]
\frac{\mathrm{d}u}{\mathrm{d}y} &= \frac{G}{\mu}y + C_1,
\label{eq:ode_int1}\\[4pt]
u(y) &= \frac{G}{2\mu}y^2 + C_1 y + C_2,
\label{eq:ode_int2}
\end{align}
where \(C_1,C_2\) are integration constants.

Apply symmetry / boundary conditions. For flow between two stationary plates located at \(y=\pm h/2\) we have no-slip
\begin{align}
u\!\left(y=\tfrac{h}{2}\right) &= 0, \label{bc1}\\
u\!\left(y=-\tfrac{h}{2}\right) &= 0. \label{bc2}
\end{align}
Subtracting the two boundary conditions eliminates \(C_2\) and yields \(C_1=0\) (symmetry implies zero shear at the centreline). Solving for \(C_2\) gives
\begin{align}
C_1 &= 0, \label{C1}\\
C_2 &= -\frac{G}{8\mu}h^2. \label{C2}
\end{align}

Substitute back to obtain the velocity profile
\begin{equation}
u(y) \;=\; \frac{G}{2\mu}\,y^2 - \frac{G}{8\mu}\,h^2
\;=\; \frac{G}{2\mu}\!\left(y^2 - \frac{h^2}{4}\right).
\label{eq:velocity_y}
\end{equation}

It is conventional to express with \(-\) sign if we set \(G=-\mathrm{d}p/\mathrm{d}x\) as the negative pressure gradient (so \(G>0\) when pressure decreases in \(x\)):
\begin{equation}
u(y) \;=\; -\frac{1}{2\mu}\left(\frac{\mathrm{d}p}{\mathrm{d}x}\right)\!\left(\frac{h^2}{4}-y^2\right).
\label{eq:velocity_y_convention}
\end{equation}

Maximum velocity at the centerline \(y=0\):
\begin{equation}
u_{\max}=u(0)= -\frac{1}{8\mu}\left(\frac{\mathrm{d}p}{\mathrm{d}x}\right) h^2.
\label{eq:umax}
\end{equation}

Volumetric flow rate per unit span (unit depth in \(z\)) is
\begin{align}
Q' &= \int_{-h/2}^{h/2} u(y)\,\mathrm{d}y
    = -\frac{1}{12\mu}\left(\frac{\mathrm{d}p}{\mathrm{d}x}\right)h^3.
\label{eq:flowrate}
\end{align}
Alternatively, in terms of \(u_{\max}\):
\begin{equation}
Q' \;=\; \frac{2}{3}u_{\max}\,h.
\label{eq:flowrate_umax}
\end{equation}

\section*{Variable definitions}
\begin{description}
  \item[\(x\)] streamwise coordinate (m).
  \item[\(y\)] wall-normal coordinate; plates at \(y=\pm h/2\) (m).
  \item[\(h\)] total channel height (distance between plates) (m).
  \item[\(u(x,y,t)\)] streamwise velocity component (m\,s\(^{-1}\)).
  \item[\(v(x,y,t)\)] wall-normal velocity component (m\,s\(^{-1}\)).
  \item[\(p(x,y,t)\)] pressure (Pa).
  \item[\(\rho\)] fluid density (kg\,m\(^{-3}\)).
  \item[\(\mu\)] dynamic viscosity (Pa\,s).
  \item[\(G\)] shorthand for \(\mathrm{d}p/\mathrm{d}x\) (Pa\,m\(^{-1}\)); take care of its sign convention.
  \item[\(Q'\)] volumetric flow rate per unit span (m\(^2\)s\(^{-1}\)).
  \item[\(u_{\max}\)] maximum (centerline) velocity (m\,s\(^{-1}\)).
\end{description}

\section*{Short answers to your conceptual questions}
\begin{itemize}
  \item \textbf{If I fix the wall temperature is it Dirichlet or Neumann?}\\
    Fixing the wall temperature is a Dirichlet boundary condition (value specified). A Neumann thermal condition would be specifying the heat flux \(\left.k\frac{\partial T}{\partial n}\right|_{\text{wall}}\).
  \item \textbf{Is ``momentum'' the convective term in Navier--Stokes?}\\
    The convective term is \((\mathbf{u}\cdot\nabla)\mathbf{u}\) and represents convective momentum transport (nonlinear advection of momentum). ``Momentum equation'' refers to the whole Navier--Stokes momentum balance, of which the convective term is one part.
  \item \textbf{When you call second-order upwind for momentum what does it change?}\\
    Second-order upwind increases the spatial accuracy of the discretisation of the convective (advection) term from first to second order. Practically this reduces numerical diffusion and gives a sharper profile for velocity/quantity gradients, at the cost of slightly larger stencil and potential small non-physical oscillations if the solution has steep discontinuities. It modifies how face values are reconstructed (quadratic/linear reconstruction instead of first-order backward).
\end{itemize}
