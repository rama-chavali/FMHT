 \chapter{Conclusion and future work}
\begin{comment}
In this work, a three dimensional ramp induced shockwave  boundary layer interactions with a compression ramp angle of $6.8^\circ$ are studied to understand the effect of sweep. The angle of the sweep used is a modest 3 degrees. This is one of the first reported simulations in the laminar region to understand these effects. The simulations are run on an NVIDIA A 100 GPGPU, and the run time of all the simulations is $\sim 50$ days. A grid size of $\sim 72$ million is needed to achieve grid independence solution. By comparing with the unswept simulations of \cite{raja} the following are observed
 
(a) the centre-line separation length was reduced by  $\sim 34\%$ when compared to the centre-line separation of unswept case. 
(b) the maximum length of the separation occurs at a distance of $\sim 10\%$ away from the symmetry plane towards the sweep direction. 
(c) No owl-like separation is observed in this work, where as the first kind was observed by \citep{raja}, and (d) the vortex that is formed in the central separation region of the unswept case is not found in the current work, which is good for reducing vibrations and oscillations in the flow field.
  
Finally to summarise, introducing small sweep angles in applications where missing is not of importance will enhance the structural stability of the vehicle and also reduces the vibrations, which in turn reduce the noise generated.
\end{comment}
% \noindent In this work, a three-dimensional ramp-induced shock-wave boundary layer interactions with a compression ramp angle of $6.8^\circ$ are studied to understand the effect of sweep. The angle of the sweep used is a modest 3 degrees. This is one of the first reported simulations in the laminar region to understand these effects. The simulations are run on an NVIDIA A100 GPGPU, and the run time of all the simulations is $\sim 50$ days. A grid size of $\sim 72$ million is needed to achieve a grid-independent solution. By comparing with the unswept simulations of \cite{raja}, the following are observed: (a) the centre-line separation length was reduced by $\sim 34\%$ when compared to the centreline separation of the unswept case, (b) the maximum length of the separation occurs at a distance of $\sim 10\%$ away from the symmetry plane towards the sweep direction, (c) No owl-like separation is observed in this work, whereas the first kind was observed by \citep{raja}, and (d) the vortex that is formed in the central separation region of the unswept case is not found in the current work, which is good for reducing vibrations and oscillations in the flow field.
%   
  
  
% \noindent Finally, to summarize, introducing small sweep angles in applications where missing is not of importance will enhance the structural stability of the vehicle and also reduce the vibrations, which in turn reduces the noise generated.
%     %symmetrical geometry and the maximum $L_s$ has moved away 12 units from the centerline. The maximum separation length wti reference is $31.1\%$. 
%  % No Owl-like separations are formed. coefficient of skin friction region showed no disturbances/ secondary humps that were previously observed in the reference \cite{raja}.The separation region in the reference case was stronger than in the swept configuration. However, the recovery of energy was greater along the centerline of the bottom wall in the swept configuration. In all other regions, the reference case showed stronger recovery compared to the swept configuration. Minimal increments in sweep angle can reduce the recirculation region which will reduce the fatigue with in the machinery\\
% 
% \noindent This research can be extended to deepen the understanding of flow physics of laminar shock-wave boundary layer interactions by
% \begin{itemize}
% 	\item Increasing the sweep angle $\phi$ 
% 	\item Increasing compression ramp angle $\theta$ 
% 	\item Changing aspect ratio of the computational domain  
% 	\item Varying the free stream Mach number  
% 	\item Confining the domain by using adiabatic wall condition along the wall-normal direction 
% \end{itemize}
